%---------------------------------------------------------------
\chapter*{Úvod}
\addcontentsline{toc}{chapter}{Úvod}
\markboth{Úvod}{Úvod}
\label{uvod}
%---------------------------------------------------------------

\setcounter{page}{1}

Zpěv je nedílnou součástí náboženských shromáždění již od jejich vzniku. Od počátku se ale věřící v náboženských shromážděních potýkají s dvěma hlavními problémy: prvním je příprava písní, druhým pak koordinace věřících a hudebního doprovodu při jejich zpěvu.

Způsob přípravy písní k zpěvu je často velmi zastaralý a způsobuje náročné přidávání nových písní a úpravu existujících. Hudební doprovod tak ztrácí motivaci k aktualizacím zpěvníku a mnohá shromáždění tedy od vzniku zpívají neustále stejné písně, což odrazuje nově příchozí členy shromáždění.

Koordinace zpěvu věřících a hudebního doprovodu se skládá ze dvou podproblémů: koordinace v rámci písně (mezi jednotlivými částmi -- refrén, sloka, přechod) a koordinace písní v rámci shromáždění (pořadí písní). První zmíněný podproblém není až tak závažný, jelikož zpěvníky obsahují vždy celý text písně -- věřící i hudební doprovod se tedy dokážou rychle zorientovat. Druhý problém již závažný je -- pokud věřící neví, kterou píseň zpívat, celý blok zpěvu písní pro něj ztrácí význam.

Vytvoření mobilní aplikace zpěvníku tak přinese nový, moderní a snadný způsob přidávání nových písní a úpravy existujících písní -- hudební doprovod tedy ušetří mnoho času, který bude moct věnovat přípravě nových písní, které přilákají nové členy shromáždění. Mobilní aplikace zpěvníku také vyřeší problém s koordinací zpěvu písní v rámci shromáždění -- věřící tak budou vždy vědět, kterou píseň mají zpívat. Dalšími problémy, jako jsou nácvik zpěvu, hudební stránka zkoušek hudebního doprovodu nebo duchovní rozvoj věřících, se má bakalářská práce zabývat nebude.

Členem náboženského shromáždění jsem již od útlého věku, a tak jsem měl téměř celý život možnost pozorovat tyto problémy. V roce 2017 jsem podnikl první krok pro digitalizaci zpěvníku, kterým bylo přepsání všech (přibližně 300) písní z papírového zpěvníku jednoho z~náboženských shromáždění do digitální podoby. Když jsem byl minulý rok osloven vedoucími kapel náboženských shromáždění s žádostí o tvorbu mobilní aplikace zpěvníku v rámci bakalářské práce, neváhal jsem dlouho a nabídku jsem přijal.

Ve své práci projdu celým procesem tvorby mobilní aplikace od sběru požadavků, analýzy fungování náboženských shromáždění a stávajících řešení koordinace zpěvu věřících a hudebního doprovodu přes návrh architektury a výběr vhodných technologií, implementaci a testování až po nasazení aplikace do obchodu pro mobilní zařízení.
